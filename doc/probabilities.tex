% Copyright (C) 2019  Stefan Vargyas
% 
% This file is part of Sha1-Prefix.
% 
% Sha1-Prefix is free software: you can redistribute it and/or modify
% it under the terms of the GNU General Public License as published by
% the Free Software Foundation, either version 3 of the License, or
% (at your option) any later version.
% 
% Sha1-Prefix is distributed in the hope that it will be useful,
% but WITHOUT ANY WARRANTY; without even the implied warranty of
% MERCHANTABILITY or FITNESS FOR A PARTICULAR PURPOSE.  See the
% GNU General Public License for more details.
% 
% You should have received a copy of the GNU General Public License
% along with Sha1-Prefix.  If not, see <http://www.gnu.org/licenses/>.

\documentclass[a4paper,9pt,leqno]{article}
\usepackage[hang]{footmisc}
\usepackage[fleqn]{amsmath}
\usepackage{relsize}
\usepackage{quoting}
\usepackage{amsfonts}
\usepackage{amssymb}
\usepackage{amsthm}
\usepackage{suffix}
\usepackage{ifpdf}
\usepackage{cite}
\usepackage{url}

\ifpdf
\pdfinfo{
/Title(Notes on Sha1-Prefix's Probabilities)
/Author(Stefan Vargyas, stvar@yahoo.com)
/Creator(LaTeX)
}
\fi

\newcommand{\vargyas}{\c{S}tefan Vargyas}

\renewcommand{\ttdefault}{pcr}
\renewcommand{\familydefault}{\sfdefault}

\addtolength{\textwidth}{1cm}
\addtolength{\evensidemargin}{-.5cm}
\addtolength{\oddsidemargin}{-.5cm}

\renewcommand{\=}{\protect\nobreakdash-\hspace{0pt}}
\renewcommand{\~}{\protect\nobreakdash--\hspace{0pt}}

\setlength{\parindent}{0pt}

\newcommand{\Shaoneprefix}{Sha1\=\!Prefix}
\newcommand{\ShaonePrefix}{\textbf{\Shaoneprefix}}

\newcommand{\ie}{i.e.}
\newcommand{\code}[1]{{\tt{#1}}}

\swapnumbers
\theoremstyle{plain}
\newtheorem{fact}{Fact}
\newtheorem{axioms}[fact]{Axioms}
\newtheorem{prop}[fact]{Proposition}
\theoremstyle{definition}
\newtheorem{defn}[fact]{Definition}
\newtheorem{notn}[fact]{Notation}
\theoremstyle{remark}
\newtheorem{rem}[fact]{Remark}

\newcommand{\impll}{\:\:\Longrightarrow\:\:}
\newcommand{\impl}{\impll}%!!!{\:\:\Rightarrow\:\:}

\newcommand{\by}[1]{{#1}}
\newcommand{\bydef}{\by{def}}

\newcommand{\eq}{=}
\newcommand{\eqby}[1]{\symbyrm{#1}{=}}
\newcommand{\eqbydef}{\eqby{\bydef}}

\newcommand{\symby}[2]{\stackrel{#1}{{#2}}}
\newcommand{\symbyrm}[2]{\symby{{\rm\scriptstyle{#1}}}{#2}}

\newcommand{\gt}{>}
\newcommand{\lt}{<}

\newcommand{\lp}{\left(}
\newcommand{\rp}{\right)}

\newcommand\Nat{\mathbb{N}}
\WithSuffix
\newcommand\Nat*{\mathbb{N}^*}

\title{Notes on \ShaonePrefix's Probabilities}
\author{\larger\vargyas\medskip\\ \tt\textbf{stvar@yahoo.com}}
\date{Jul 15, 2019}

\newlength\citespace
\setlength\citespace{.3pt}
\renewcommand{\citeleft}{\mbox{\textbf{[}\hspace{\citespace}}}
\renewcommand{\citeright}{\mbox{\hspace{\citespace}\textbf{]}}}
\renewcommand{\citeform}[1]{\textbf{#1}}

\makeatletter
\renewcommand\@biblabel[1]{\textbf{[#1]}}%
\makeatother
\newenvironment{bbibliography}[1]{%
    \begin{thebibliography}{\textbf{#1}}%
    \DeclareUrlCommand\url{\urlstyle{tt}\vspace{\parskip}\\}
}{%
    \end{thebibliography}%
}

\begin{document}
\maketitle

\section{The \ShaonePrefix\ Program}

\ShaonePrefix\ is a solution program to the following {\it SHA1 prefix problem}:

\begin{quoting}
Given an input UTF\=8 string and a {\it difficulty} number
$n \in \Nat$ with $1 \le n \le 9$, do find an UTF\=8 string that
appended to the input string would produce a SHA1 digest of
which hexadecimal representation has its leftmost $n$ digits
all equal to $0$.
\end{quoting}

\ShaonePrefix\ is structured on two tiers:

\begin{itemize}

\item \code{sha1-prefix} --- obtained from \code{sha1-prefix.c} and
a few other source files --- is the main program tackling the core of
the SHA1 prefix problem. It gets the input string (not necessarily
UTF\=8) from \code{stdin} or from a named file and the difficulty
number as a command line argument. \code{sha1-prefix} will run until it
finds the first suffix string that satisfies the specified prefix condition.
It can be told to terminate cleanly the execution when signaled with
\code{SIGHUP} if \code{`-e|--sighup-exits'} was passed on the invoking
command line.

\item \code{sha1-prefix.sh} is a quite involved \code{bash} shell
script that drives the main program's execution. The shell script's
main use case is that of running \code{sha1-prefix} in series
controlled by time outs.
\end{itemize}

Bellow we will present simple theoretical arguments that founds the
approach taken. Under reasonably acceptable assumptions, we show that
running \code{sha-prefix} in series increases \ShaonePrefix's probability
of success producing required outcome.

\section{Mathematical Evaluations}

The basic theoretical assumption we're considering hereafter
is that successive runs of \code{sha1-prefix} by \code{sha1-prefix.sh}
are to be modeled as Bernoulli trials \cite{wiki:bernoullitrial} with
constant or, by case, non-constant probability of success.
%
\begin{fact}\label{binomial:probability}
If the random variable $X$ follows the binomial distribution
with parameters $n \in \Nat*$ and $0 \le p \le 1$, then the
probability of getting \emph{at least one success} in $n$ trials
is given by $1 - (1 - p)^n$.
\end{fact}

\begin{proof}
As per \cite{wiki:binomialdist}, the probability of exactly
$k$ successes in $n$ trials is given by:
%
\begin{equation*}
\Pr(X \eq k) = {n\choose k} p^k (1-p)^{n-k} 
\end{equation*}
%
Then follows easily that:
%
\begin{align*}
\Pr(X \ge 1) & = 1 - \Pr(X \lt 1) \\
             & = 1 - \Pr(X \eq 0) \\
             & = 1 - {n\choose 0} p^0 (1-p)^{n - 0} \\
             & = 1 - (1 - p)^n
\end{align*}
%
\end{proof}

\begin{fact}
If $0 \lt p \lt 1$,
for $B_n \eqbydef 1 - (1 - p)^n$, it holds true that
$p \lt B_n \lt B_{n+1}$, for all $n \in \Nat*$.
\end{fact}

\begin{proof}
%
We have that:
%
\begin{align*}
0 \lt p \lt 1 & \impl 0 \lt 1 - p \lt 1 \\
              & \impl (1 - p)^{n-1} \lt 1 \\
              & \impl (1 - p)^n \lt 1 - p \\
              & \impl p \lt B_n
\end{align*}
%
Moreover:
%
\begin{align*}
0 \lt p \lt 1 & \impl 0 \lt 1 - p \lt 1 \\
              & \impl (1 - p)^{n+1} \lt (1 - p)^n \\
              & \impl B_n \lt B_{n+1}
\end{align*}
%
\end{proof}

\begin{fact}
If the random variable $X$ follows the Poisson's binomial distribution
with parameters $n \in \Nat*$ and $0 \le p_i \le 1$, for $1 \le i \le n$,
then the probability of getting \emph{at least one success} in $n$ trials
is given by $1 - \prod_{i=1}^n (1 - p_i)$.
\end{fact}

\begin{proof}
As per \cite{wiki:poissonbinomialdist}, the probability of exactly
$0$ successes in $n$ trials is given by:
%
\begin{equation*}
\Pr(X \eq 0) = \prod_{i=1}^n (1 - p_i) 
\end{equation*}
%
Then:
%
\begin{align*}
\Pr(X \ge 1) & = 1 - \Pr(X \lt 1) \\
             & = 1 - \Pr(X \eq 0) \\
             & = 1 - \prod_{i=1}^n (1 - p_i)
\end{align*}
%
\end{proof}

\begin{fact}
If $0 \lt p_i \lt 1$ for all $i \in \Nat$, $1 \le i \le n$,
for $P_n \eqbydef 1 - \prod_{i=1}^n(1 - p_i)$, it holds true that
$p_i \lt P_n \lt P_{n+1}$, for all $i \in \Nat$, $1 \le i \le n$,
and $n \in \Nat*$.
\end{fact}

\begin{proof}
%
For arbitrary but fixed $i \in \Nat$, $1 \le i \le n$, we have that:
%
\begin{align*}
0 \lt p_i \lt 1 & \impl 0 \lt 1 - p_i \lt 1 \\
                & \impl \prod_{k=1}^n (1 - p_k) \lt 1 - p_i \\
                & \impl p_i \lt P_n
\end{align*}
%
Moreover:
%
\begin{align*}
0 \lt p_i \lt 1 & \impl 0 \lt 1 - p_i \lt 1 \\
                & \impl \prod_{k=1}^{n+1} (1 - p_k)
                       \eq \lp \prod_{k=1}^n (1 - p_k) \rp \cdot (1 - p_{n+1}) 
                       \lt \prod_{k=1}^n (1 - p_k) \\
                & \impl P_n \lt P_{n+1}
\end{align*}
%
\end{proof}

\begin{bbibliography}{0}
%
\bibitem{wiki:bernoullitrial}
    Bernoulli Trial
    \url{https://en.wikipedia.org/wiki/Bernoulli_trial}
%
\bibitem{wiki:binomialdist}
    Binomial Distribution
    \url{https://en.wikipedia.org/wiki/Binomial_distribution}
%
\bibitem{wiki:poissonbinomialdist}
    Poisson Binomial Distribution
    \url{https://en.wikipedia.org/wiki/Poisson_binomial_distribution}
%
\end{bbibliography}
%
\let\thefootnote\relax%
\footnotetext{%
\renewcommand{\Shaoneprefix}{Sha1-Prefix}%
\vspace{5pt}

Copyright \copyright\ 2019\hspace{1em}\textbf{\vargyas}

This file is part of \ShaonePrefix.

\ShaonePrefix\ is free software: you can redistribute it and/or modify
it under the terms of the GNU General Public License as published by
the Free Software Foundation, either version 3 of the License, or
(at your option) any later version.

\ShaonePrefix\ is distributed in the hope that it will be useful,
but WITHOUT ANY WARRANTY; without even the implied warranty of
MERCHANTABILITY or FITNESS FOR A PARTICULAR PURPOSE.  See the
GNU General Public License for more details.

You should have received a copy of the GNU General Public License
along with \ShaonePrefix.  If not, see \url{http://www.gnu.org/licenses/}.}
%
\end{document}


